%...% operators

%>% has no builtin meaning but the user (or a package) is free to define operators
of the form %whatever% in any way they like. For example, this function will
return a string consisting of its left argument followed by a comma and space 
and then it's right argument.

Differences between %.% (dplyr) and %>% (magrittr)
Differences include

    you can use a . as placeholder for the left-hand side, e.g.

     iris %>% plot(Sepal.Length ~ Sepal.Width, data = .)

    %>% respects (rhs), e.g.

     1:10 %>% (call("sum"))
     1:10 %>% (function(x) x^2 + 2*x) 


Practical: Introduction to Phylogenetics in R
    https://eeob-macroevolution.github.io/Practicals/Intro_to_Phylo/intro_to_phylo.html

%in% can be used to verify if an item is present in a list, vector or matrix

matrix indexing
    - matrix[i,] access the ith row in a matrix
    - matrix[,j] accessn the jth column in a matrix
    - matrix[c(i,j),] or matrix[,c(i,j)] access rows or columns i and j

matrix modifications
    - cbind to add an additional column
    - rbind to add an additional row
    - negative indexing to remove columns, rows or elements

matrices are limited to 2D in R, array is a more generalized construct that allows for N dimensional tensors
    array(valVec, dim=(nrow, ncol, ndepth,...))

data.frame
    - data.frame(col1Name=c(val11, val12,...),
                 col2Name=c(val21, val22,...),
                 ...,
                 colnName=c(valn1, valn2,...))
    - accessing columns 
        - indexing df[i]
        - colname indexing
            - df[["colName"]]
            - df$colnName

Factors
    - usefull when dealing with categorical data
    - have a fixed and known set of values

R plots 
    - 